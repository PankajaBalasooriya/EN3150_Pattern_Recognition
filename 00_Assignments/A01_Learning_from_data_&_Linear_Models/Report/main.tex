\documentclass[12pt]{article}

% ============================================================================
% PACKAGE IMPORTS - Generally no need to modify unless adding new features
% ============================================================================
\usepackage{graphicx}
\usepackage{float}
\usepackage{caption}
\usepackage{subcaption}
\usepackage{geometry}
\geometry{a4paper, margin=1in}
\usepackage{amsmath, amsfonts, amssymb}
\usepackage{booktabs, array}
\usepackage{fancyhdr}
\usepackage{titlesec}
\usepackage{listings}
\usepackage{xcolor}
\usepackage[hidelinks]{hyperref}
\usepackage{enumitem}
\usepackage{cite}

% ============================================================================
% CUSTOMIZATION SECTION - EDIT THESE VARIABLES FOR YOUR ASSIGNMENT
% ============================================================================
% Replace the values below with your specific assignment details:

\newcommand{\universityname}{University of Moratuwa}                    % Your university name
\newcommand{\departmentname}{Department of Electronic and Telecommunication Engineering}  % Your department
\newcommand{\modulecode}{EN3150}                                        % Course/module code
\newcommand{\modulename}{Pattern Recognition}                    % Course/module name
\newcommand{\assignmentname}{Learning from data and related challenges and linear models}  % Your assignment title


% Page setup
% Page margins - adjust as needed (left, right, top, bottom)
\geometry{left=2.5cm, right=2.5cm, top=2.5cm, bottom=2.5cm}

% Header and footer setup
\pagestyle{fancy}
\fancyhf{}                        % Clear all header and footer fields
\fancyhead[L]{\modulecode}      % Left header: Module code and name
\fancyhead[R]{\assignmentname}               % Right header: Assignment name
\fancyfoot[C]{\thepage}                      % Center footer: Page number


% ============================================================================
% DOCUMENT CONTENT STARTS HERE
% ============================================================================
\begin{document}

% ============================================================================
% TITLE PAGE - Modify student information below
% ============================================================================
\begin{titlepage}
    \centering
    \vspace*{1cm}

    {\huge\textbf{\universityname}}\\[1cm]

    {\Large\textbf{\departmentname}}\\[0.5cm]

    \includegraphics[width=0.5\textwidth]{resources/University_of_Moratuwa_logo.png}\\[1cm]

    {\large\textbf{\modulecode – \modulename}}\\[1cm]

    {\LARGE\textbf{\assignmentname}}\\[1cm]
    
    \vspace{0.3cm}
    
    % ========================================================================
    % STUDENT INFORMATION - CHOOSE ONE OPTION BELOW
    % ========================================================================
    
    % OPTION 1: FOR GROUP ASSIGNMENTS (Comment out Option 2 if using this)
    % \textbf{Group Members:}\\
    % \begin{tabular}{ll}
    %     Name 1 & Index Number 1 \\          % Replace with actual names and index numbers
    %     Name 2 & Index Number 2 \\          % Add more rows if needed
    %     Name 3 & Index Number 3 \\          % Remove rows if fewer group members
    %     % Name 4 & Index Number 4 \\        % Uncomment and add more rows as needed
    % \end{tabular}\\[1cm]
    
    % OPTION 2: FOR INDIVIDUAL ASSIGNMENTS (Comment out Option 1 if using this)
    % Uncomment the lines below and comment out the "Group Members" section above
    \textbf{Submitted by:}\\
    \begin{tabular}{ll}
       Balasooriya B.A.P.I. & 220054N \\
       % \textbf{Index Number:} & 220054N \\
    \end{tabular}\\[1cm]
    
    % Automatic date - will show today's date when compiled
    \textbf{Date:} \today

    \vfill
\end{titlepage}

% ============================================================================
% TABLE OF CONTENTS - Automatically generated
% ============================================================================
\newpage
\tableofcontents
\newpage


% ============================================================================
% MAIN CONTENT - REPLACE WITH YOUR ACTUAL ASSIGNMENT CONTENT
% ============================================================================

% Use \section{} for main sections, \subsection{} for subsections, 
% and \subsubsection{} for further subdivisions

\section{Linear regression impact on outliers}







\newpage
\section{Sample Section Title}
This is a sample paragraph introducing the section content.  
You can describe the overall idea, provide background information, and define the scope of discussion\cite{rayleigh1896sound}.
% Note: \cite{} is used for referencing. Make sure to add your references to references.bib file

\subsection{Sample Subsection Title}
This is an example of a subsection.  
You can use bullet points or paragraphs to elaborate:
\begin{itemize}
    \item First key idea in this subsection.
    \item Second key idea with additional explanation.
    \item Third point highlighting a detail.
\end{itemize}

\subsubsection{Sample Subsubsection Title}
Here is an example of a subsubsection.  
It is used when you need an extra level of detail.  
For example:
\begin{enumerate}
    \item Step-by-step explanation.
    \item Specific data or figures.
    \item Important notes.
\end{enumerate}

\section{Another Sample Section}
This section can be used for additional content, analysis, or discussion.  
You can include:
\begin{itemize}
    \item Figures (see Figure~\ref{fig:sample_fig})
    \item Tables
    \item Equations
\end{itemize}

% ============================================================================
% FIGURE EXAMPLE - How to include images and figures
% ============================================================================
\begin{figure}[H]       % [H] forces the figure to appear exactly here
    \centering
    % Uncomment the line below and replace with your actual image file
    %\includegraphics[width=0.5\textwidth]{resources/sample_image.png}

    % The line below creates a placeholder box - remove it when using real images
    \fbox{\rule{0pt}{2in}\rule{2in}{0pt}}
    \caption{Example placeholder figure.}
    \label{fig:sample_fig}
\end{figure}

% ============================================================================
% CONCLUSION SECTION
% ============================================================================
\section*{Conclusion}   % * creates unnumbered section
\addcontentsline{toc}{section}{Conclusion}
This is the conclusion section.  
Summarize the main points covered, key takeaways, and any recommendations for future work.

% ============================================================================
% REFERENCES/BIBLIOGRAPHY SECTION
% ============================================================================
% Make sure you have a 'references.bib' file with your bibliography entries
% Example entry in references.bib:
% @article{rayleigh1896sound,
%   title={The theory of sound},
%   author={Rayleigh, Lord},
%   year={1896},
%   publisher={Macmillan}
% }
\newpage
\addcontentsline{toc}{section}{References}    % Add References to table of contents
\bibliographystyle{IEEEtran}                  % IEEE citation style (common in engineering)
\bibliography{references}                      % References from references.bib file

\end{document}

% ============================================================================
% ADDITIONAL HELPFUL TIPS:
% ============================================================================
%
% 1. ADDING IMAGES:
%    - Create a 'resources' folder in your project directory
%    - Place image files there (PNG, JPG, PDF formats work well)
%    - Use \includegraphics[width=0.5\textwidth]{resources/your_image.png}
%
% 2. MATHEMATICS:
%    - Inline math: $x = y + z$
%    - Display math: \[x = y + z\]
%    - Numbered equations: \begin{equation} x = y + z \end{equation}
%
% 3. TABLES:
%    \begin{table}[H]
%    \centering
%    \begin{tabular}{|c|c|c|}
%    \hline
%    Header 1 & Header 2 & Header 3 \\
%    \hline
%    Data 1   & Data 2   & Data 3   \\
%    \hline
%    \end{tabular}
%    \caption{Your table caption}
%    \label{tab:your_label}
%    \end{table}
%
% 4. CROSS-REFERENCES:
%    - Reference figures: Figure~\ref{fig:your_label}
%    - Reference tables: Table~\ref{tab:your_label}
%    - Reference sections: Section~\ref{sec:your_label}
%
% ============================================================================
